\documentclass{llncs}

\setlength{\textwidth}{13.5cm}
\advance\evensidemargin by -.65cm
\advance\oddsidemargin by -.65cm

\usepackage{fleqn}
\usepackage{listings}
\usepackage{math}
\usepackage{amsmath}
\usepackage{latexsym}
\usepackage{bcprules}
\usepackage[T1]{fontenc}

% Prooftree formatting
\usepackage{prooftree}

\usepackage[bookmarks=true]{hyperref}
\usepackage{bookmark}

\usepackage{xcolor}

% verbfilter stuff
\newcommand{\prog}[1]{{\sl #1}}
\newenvironment{program}[1][10.5]
  {\fontsize{#1}{13.6}\tt\begin{tabbing}\hspace*{0.5\parindent}\=\+\kill}
  {\end{tabbing}\noindent}
\newcommand{\blockcomment}[1]{{\color{grayPoint3}#1}}
\newcommand{\linecomment}{\color{grayPoint3}}
\newcommand{\grey}{\color{grey}}

%\newenvironment{program}{\ \ \ \ \begin{minipage}{\textwidth}\renewcommand{\baselinestretch}{1.0}\sl\begin{tabbing}}{\end{tabbing}\end{minipage}}
\newcommand{\vem}{\bfseries}
\newcommand{\quotedstring}[1]{{#1}}
\newcommand{\typename}[1]{{#1}}
\newcommand{\literal}[1]{{#1}}

% comments and notes
\newcommand{\comment}[1]{}
%\newcommand{\note}[1]{{\bf $\clubsuit$ #1 $\spadesuit$}}

% figures
\newcommand{\figurebox}[1]
        {\fbox{\begin{minipage}{\textwidth} #1 \medskip\end{minipage}}}
\newcommand{\twofig}[3]
        {\begin{figure*}[t]#3\ \hrulefill\ 
        \caption{\label{#1}#2}\end{figure*}}
\newcommand{\boxfig}[3]
        {\begin{figure*}\figurebox{#3\caption{\label{#1}#2}}\end{figure*}}
\newcommand{\figref}[1]
        {Figure~\ref{#1}}

% typing rules (not used here)
\newcommand{\ttag}[1]{\mbox{\textsc{\small(#1)}}}
\newcommand{\infer}[3]{\mbox{#1 }\ba{c} #2 \\ \hline #3 \ea}
\newcommand{\irule}[2]{{\renewcommand{\arraystretch}{1.2}\ba{c} #1 
                        \\ \hline #2 \ea}}
\newlength{\trulemargin}
\newlength{\trulewidth}
\newlength{\srulewidth}
\setlength{\trulemargin}{1.75cm}
\setlength{\trulewidth}{83.7mm}
\setlength{\srulewidth}{6.0cm}
\newenvironment{trules}{$\vspace{0.5em}\ba{p{\trulemargin}@{~}p{\trulewidth}@{~}p{\trulemargin}}}{\ea$}
\newenvironment{srules}{$\vspace{0.5em}\ba{p{\trulemargin}@{~}p{\srulewidth}}}{\ea$}
\newcommand{\laxiom}[2]{\ttag{#1} & $ #2 \hfill\ }
\newcommand{\raxiom}[2]{\hfill #2 $& \hfill \ttag{#1}}
\newcommand{\caxiom}[2]{\ttag{#1} & $\hfill #2 \hfill $& \ }
\newcommand{\lrule}[3]{\laxiom{#1}{\irule{#2}{#3}}}
\newcommand{\rrule}[3]{\raxiom{#1}{\irule{#2}{#3}}}
\newcommand{\crule}[3]{\caxiom{#1}{\irule{#2}{#3}}}
\newcommand{\lsrule}[3]{\lsaxiom{#1}{\irule{#2}{#3}}}
\newcommand{\rsrule}[3]{\rsaxiom{#1}{\irule{#2}{#3}}}
\newcommand{\nl}{\end{trules}\\[0.5em] \begin{trules}}
\newcommand{\snl}{\end{srules}\\[0.5em] \begin{srules}}

% commas and semicolons
\newcommand{\comma}{,\,}
\newcommand{\commadots}{\comma \ldots \comma}
\newcommand{\semi}{;\mbox{;};}
\newcommand{\semidots}{\semi \ldots \semi}

% spacing
\newcommand{\gap}{\quad\quad}
\newcommand{\biggap}{\quad\quad\quad}
\newcommand{\nextline}{\\ \\}
\newcommand{\htabwidth}{0.5cm}
\newcommand{\tabwidth}{1cm}
\newcommand{\htab}{\hspace{\htabwidth}}
\newcommand{\tab}{\hspace{\tabwidth}}
\newcommand{\linesep}{\ \hrulefill \ \smallskip}

% math stuff
\newenvironment{myproof}{{\em Proof:}}{$\Box$}
\newenvironment{proofsketch}{{\em Proof Sketch:}}{$\Box$}
\newcommand{\Case}{{\em Case\ }}

% make ; a delimiter in math mode
% \mathcode`\;="8000 % Makes ; active in math mode
% {\catcode`\;=\active \gdef;{\;}}
% \mathchardef\semicolon="003B

% reserved words
\newcommand{\mathem}{\bf}

% brackets
\newcommand{\set}[1]{\{#1\}}
\newcommand{\sbs}[1]{\lquote #1 \rquote}

% arrays
\newcommand{\ba}{\begin{array}}
\newcommand{\ea}{\end{array}}
\newcommand{\bda}{\[\ba}
\newcommand{\eda}{\ea\]}
\newcommand{\ei}{\end{array}}
\newcommand{\bcases}{\left\{\begin{array}{ll}}
\newcommand{\ecases}{\end{array}\right.}

% \cal ids
\renewcommand{\AA}{{\cal A}}
\newcommand{\BB}{{\cal B}}
\newcommand{\CC}{{\cal C}}
\newcommand{\DD}{{\cal D}}
\newcommand{\EE}{{\cal E}}
\newcommand{\FF}{{\cal F}}
\newcommand{\GG}{{\cal G}}
\newcommand{\HH}{{\cal H}}
\newcommand{\II}{{\cal I}}
\newcommand{\JJ}{{\cal J}}
\newcommand{\KK}{{\cal K}}
\newcommand{\LL}{{\cal L}}
\newcommand{\MM}{{\cal M}}
\newcommand{\NN}{{\cal N}}
\newcommand{\OO}{{\cal O}}
\newcommand{\PP}{{\cal P}}
\newcommand{\QQ}{{\cal Q}}
\newcommand{\RR}{{\cal R}}
\newcommand{\TT}{{\cal T}}
\newcommand{\UU}{{\cal U}}
\newcommand{\VV}{{\cal V}}
\newcommand{\WW}{{\cal W}}
\newcommand{\XX}{{\cal X}}
\newcommand{\YY}{{\cal Y}}
\newcommand{\ZZ}{{\cal Z}}

% misc symbols
\newcommand{\dhd}{\!\!\!\!\!\rightarrow}
\newcommand{\Dhd}{\!\!\!\!\!\Rightarrow}
% in case it's not yet defined:
\providecommand{\ts}{}
\renewcommand{\ts}{\,\vdash\,}
\newcommand{\la}{\langle}
\newcommand{\ra}{\rangle}
\newcommand{\eg}{{\em e.g.}}

% misc identifiers
\newcommand{\dom}{\mbox{\sl dom}}
\newcommand{\fn}{\mbox{\sl fn}}
\newcommand{\bn}{\mbox{\sl bn}}
\newcommand{\sig}{\mbox{\sl sig}}
\newcommand{\IF}{\mbox{\mathem if}}
\newcommand{\OTHERWISE}{\mbox{\mathem otherwise}}
\newcommand{\expand}{\prec}
\newcommand{\weakexpand}{\prec:}
\newcommand{\anyexpand}{\prec}
\newcommand{\spcomma}{~,~}

%\newcommand{\inst}{\mbox{\mathem inst}}
\newcommand{\trans}[1]{\la\!\la#1\ra\!\ra}
% in case it's not yet defined:
\providecommand{\remark}{}
\renewcommand{\remark}[1]{{\bf $\clubsuit$ #1 $\spadesuit$}}
\newcommand{\todo}[1]{\remark{to do: #1}}
%\newcommand{\J}{\justifies}
%\newcommand{\U}{\using}

% names
\newcommand{\Scala}{\mbox{\textsc{Scala}}}
\newcommand{\Java}{\mbox{\textsc{Java}}}

%\renewcommand\textfraction{.05}
%\renewcommand\floatpagefraction{.9}
%\renewcommand\topfraction{.8}

%%%%%%%%%%%%%%%%%%%%%%%%%%%%%%%%%%%%%%%
%   Language abstraction commands     %
%%%%%%%%%%%%%%%%%%%%%%%%%%%%%%%%%%%%%%%

%% Relations
% Subtype 
\newcommand{\sub}{<:}
% Type assignment
\newcommand{\typ}{:}
\newcommand{\approxtyp}{:_{<:}}
% reduction
\newcommand{\reduces}{\;\rightarrow\;}
\newcommand{\resolves}{\Downarrow}
\newcommand{\irresolves}{\Updownarrow}
\newcommand{\optresolves}{\Updownarrow}
% well-formedness
\newcommand{\wf}{\;\mbox{\textbf{wf}}}
% non-membership
\newcommand{\hasnt}{\not\owns}

%% Operators
% Type selection
\newcommand{\tsel}{\#}
% Function type
\newcommand{\tfun}{\rightarrow}
\newcommand{\dfun}[3]{(#1\!:\!#2) \Rightarrow #3}
% Conjunction
\newcommand{\tand}{\wedge}
% Disjunction
\newcommand{\tor}{\vee}
% Singleton type suffix
\newcommand{\sing}{.\textbf{type}}
% Existential type
\newcommand{\exTyp}[3]{\exists(#1: #2)#3}

%% Syntax
% Header for typing rules
\newcommand{\judgement}[2]{{\bf #1} \hfill #2}
% Refinement
\newcommand{\refine}[2]{\left\{#1 \Rightarrow #2 \right\}}
% Field definitions
\newcommand{\ldefs}[1]{\left\{#1\right\}}
% Member sequences
\newcommand{\seq}[1]{\overline{#1}}
% Lambda
\newcommand{\dabs}[3]{(#1\!:\!#2)\Rightarrow #3}
\newcommand{\abs}[3]{\lambda #1\!:\!#2.#3}
% Application
\newcommand{\app}[2]{#1\;#2}
\newcommand{\trmcall}[3]{#1.#2(#3)}
% Method Application
\newcommand{\mapp}[3]{#1.#2(#3)}
% Substitution
\newcommand{\subst}[3]{[#1/#2]#3}
% Object creation
\newcommand{\trmnew}[3]{\textbf{val }#1 = \textbf{new }\{#2\} ;\; #3}
\newcommand{\deftyp}[2]{#1 = #2}
\newcommand{\defmtd}[5]{#1(#2 : #3): #4 = #5}
\newcommand{\anfmapp}[5]{\textbf{val }#1 = {#2.#3(#4)} ;\; #5}
\newcommand{\anfmexe}[5]{\textbf{val }#1 = {#2.#3\ldots\;#4} ;\; #5}
%\renewcommand{\new}[3]{#1 \leftarrow #2 \,\textbf{in}\, #3}
% Field declaration
\newcommand{\Ldecl}[3]{#1 : #2..#3}%{#1 \operatorname{>:} #2 \operatorname{<:} #3}
\newcommand{\ldecl}[2]{#1 : #2}
\newcommand{\mdecl}[3]{#1 : #2 \tfun #3}
% Top and Bottom
\newcommand{\Top}{\top}%{\textbf{Top}}
\newcommand{\Bot}{\bot}%\textbf{Bot}}
% Environment extension
%\newcommand{\envplus}[1]{\uplus \{ #1 \}}
\newcommand{\envplus}[1]{, #1}
% Reduction
\newcommand{\reduction}[4]{#1 \operatorname{|} #2 \reduces #3 \operatorname{|} #4}


%%%%%%%%%%%%%%%%%%%%%%%%
%%%% BEGIN DOCUMENT %%%%
%%%%%%%%%%%%%%%%%%%%%%%%

\begin{document}
\thispagestyle{plain}
\pagestyle{plain}
\mainmatter

\title{[DRAFT] The exDOT Calculus}
\author{Samuel Gruetter, Nada Amin, Martin Odersky}
\institute{EPFL}

\maketitle
\sloppy
\newcommand{\lindent}{\hspace{-4mm}}

%\newcommand{\highlight}[1]{\fcolorbox{lightgray}{lightgray}{#1}}

\newcommand{\highlight}[1]{\colorbox{lightgray}{#1}}

\newcommand{\HIGHLIGHT}[1]{\colorbox{lightgray}{$#1$}}

This document presents gDOT, and the additions made by exDOT are \highlight{highlighted} in gray.

\begin{figure}
\figurebox{
%\renewcommand{\baselinestretch}{0.95}
\pdfbookmark[0]{Syntax}{syntax}
{\bf Syntax}\medskip
    
$\ba{l@{\hspace{0.2mm}}|@{\hspace{0.2mm}}l}
\ba[t]{l@{\hspace{10mm}}l}
x, y, z    & \lindent{\mbox{Variable}} \\[0.2em]
l ::=      & \lindent{\mbox{Label}} \\
\gap L     & \mbox{Type label}\\
\gap m     & \mbox{Method label}\\[0.2em]
t, u ::=      & \lindent{\mbox{Term}} \\
\gap x     & \mbox{variable} \\
\gap \trmnew x {\seq{d}} t & \mbox{new instance} \\
%\gap \highlight{$\trmcall t m u$} & \mbox{method invocation} \\[0.2em]
\gap \trmcall t m u & \mbox{method invocation} \\[0.2em]
d ::= & \lindent{\mbox{Initialization}} \\
\gap \deftyp L T        & \mbox{field init.}\\
\gap \defmtd m x T U u  & \mbox{method init.}\\[0.2em]
\Gamma ::= \seq{x \typ T} & \lindent\mbox{Environment} \\[0.2em]
s      ::= \seq{x \mapsto \{\seq{d}\}} & \lindent\mbox{Store} \\
\ea
&

\ba[t]{l@{\hspace{10mm}}l}
S,T,U,V,W ::= & \lindent\mbox{Type}\\
\gap \Top  & \mbox{top type} \\
\gap \Bot  & \mbox{bottom type} \\
\gap \{D\} & \mbox{one-member record} \\
\gap x.L & \mbox{type selection} \\
\gap T \tand T & \mbox{intersection type} \\
\gap T \tor T  & \mbox{union type} \\[0.2em]
\gap \highlight{$x\sing$}    & \mbox{singleton type} \\[0.3em]
\gap \highlight{$\exTyp x T U$} & \mbox{existential type} \\
D ::= & \lindent\mbox{Declaration} \\
\gap \Ldecl {L} S U & \mbox{abstract type decl.} \\
\gap \mdecl m S U & \mbox{method declaration}
\ea
\ea$
}
%\caption{The DOT Calculus : Syntax}\label{dot-syntax}
\end{figure}


\begin{figure}
\figurebox{
\pdfbookmark[0]{Reduction}{reduction}
{\bf Reduction}\hfill\fbox{$\reduction t s {t'} {s'}$}

\infrule[\textsc{red-call}]
{x \mapsto \ldefs{\seq{\deftyp L W}\;\seq{\defmtd m z T U u}} \in s}
{\reduction {\trmcall x {m_i} y} s {\subst y {z_i} {u_i}} s}

\infrule[\textsc{red-new}]
{z \notin \dom(s)}
{\reduction {\trmnew x {\seq{d}} t} s
            {\subst z x t} {s \envplus{z \mapsto {\subst z x {\{\seq{d}\}}}}}}

\vspace{0.5em}

\begin{multicols}{2}

\infrule[\textsc{red-call-1}]
{\reduction {t} s {t'} {s'}}
{\reduction {\trmcall t m u} s {\trmcall {t'} m u} {s'}}

\infrule[\textsc{red-call-2}]
{\reduction {u} s {u'} {s'}}
{\reduction {\trmcall x m u} s {\trmcall x m {u'}} {s'}}

\end{multicols}

}
%\caption{The DOT Calculus : Reduction}\label{dot-red}
\end{figure}

\begin{figure}
\figurebox{
{\bf Declaration intersection}\hfill\fbox{$intersect(D_1, D_2) = D_3$}

\begin{multicols}{2}

\infrule
{D_1 = (\Ldecl L {S_1} {U_1}) ~~~~~~~ D_2 = (\Ldecl L {S_2} {U_2})}
{intersect(D_1, D_2) = (\Ldecl L {S_1 \tor S_2~} {~U_1 \tand U_2})}

\infrule
{D_1 = (\mdecl m {S_1} {U_1}) ~~~~~~~ D_2 = (\mdecl m {S_2} {U_2})}
{intersect(D_1, D_2) = (\mdecl m {S_1 \tor S_2~} {~U_1 \tand U_2})}

\end{multicols}

\linesep

{\bf Declaration union}\hfill\fbox{$union(D_1, D_2) = D_3$}

\begin{multicols}{2}

\infrule
{D_1 = (\Ldecl L {S_1} {U_1}) ~~~~~~~ D_2 = (\Ldecl L {S_2} {U_2})}
{union(D_1, D_2) = (\Ldecl L {S_1 \tand S_2~} {~U_1 \tor U_2})}

\infrule
{D_1 = (\mdecl m {S_1} {U_1}) ~~~~~~~ D_2 = (\mdecl m {S_2} {U_2})}
{union(D_1, D_2) = (\mdecl m {S_1 \tand S_2~} {~U_1 \tor U_2})}

\end{multicols}

}
\end{figure}

%%%%%%%%%%%%%%%%%%%%%%%%%%%%

\begin{figure}
\figurebox{
\pdfbookmark[0]{Membership}{membership}
{\bf Membership}\hfill{\fbox{$\Gamma \ts T \ni D$}}

\begin{multicols}{2}

\infrule[\textsc{$\Bot$-$\ni$-typ}]
{}
{\Gamma \ts \Bot \ni (\Ldecl L \Top \Bot)}

\infrule[\textsc{$\Bot$-$\ni$-mtd}]
{}
{\Gamma \ts \Bot \ni (\mdecl m \Top \Bot)}

\infrule[\textsc{rcd-$\ni$}]
{}
{\Gamma \ts \{D\} \ni D}

\infrule[\textsc{sel-$\ni$}]
{(x: T) \in \Gamma \\
 \Gamma \ts T \ni (\Ldecl L S U) \\
 \Gamma \ts U \ni D}
{\Gamma \ts x.L \ni D}

\infrule[\textsc{$\tand$-$\ni$-1}]
{\Gamma \ts T_1 \ni D \\
 \Gamma \ts T_2 \hasnt label(D)}
{\Gamma \ts T_1 \tand T_2 \ni D}

\infrule[\textsc{$\tand$-$\ni$-2}]
{\Gamma \ts T_2 \ni D \\
 \Gamma \ts T_1 \hasnt label(D)}
{\Gamma \ts T_1 \tand T_2 \ni D}

\infrule[\textsc{$\tand$-$\ni$-12}]
{\Gamma \ts T_1 \ni D_1 ~~~~ \Gamma \ts T_2 \ni D_2}
{\Gamma \ts T_1 \tand T_2 \ni intersect(D_1, D_2)}

\infrule[\textsc{$\tor$-$\ni$}]
{\Gamma \ts T_1 \ni D_1 ~~~~ \Gamma \ts T_2 \ni D_2}
{\Gamma \ts T_1 \tor T_2 \ni union(D_1, D_2)}

\end{multicols}

\linesep
   
\begin{multicols}{2}[\judgement{Non-membership}{\fbox{$\Gamma \ts T \hasnt l$}}]

\infrule[\textsc{$\Top$-$\hasnt$}]
{}
{\Gamma \ts \Top \hasnt l}

\infrule[\textsc{rcd-$\hasnt$}]
{l \neq label(D)}
{\Gamma \ts \{ D \} \hasnt l}

\infrule[\textsc{sel-$\hasnt$}]
{(x: T) \in \Gamma \\
 \Gamma \ts T \ni (\Ldecl L S U) \\
 \Gamma \ts U \hasnt l}
{\Gamma \ts x.L \hasnt l}

\infrule[\textsc{$\tand$-$\hasnt$}]
{\Gamma \ts T_1 \hasnt l ~~~~ \Gamma \ts T_2 \hasnt l}
{\Gamma \ts T_1 \tand T_2 \hasnt l}

\infrule[\textsc{$\tor$-$\hasnt$-1}]
{\Gamma \ts T_1 \ni D \\
 \Gamma \ts T_2 \hasnt label(D)}
{\Gamma \ts T_1 \tor T_2 \hasnt label(D)}

\infrule[\textsc{$\tor$-$\hasnt$-2}]
{\Gamma \ts T_2 \ni D \\
 \Gamma \ts T_1 \hasnt label(D)}
{\Gamma \ts T_1 \tor T_2 \hasnt label(D)}

\infrule[\textsc{$\tor$-$\hasnt$-12}]
{\Gamma \ts T_1 \hasnt l ~~~~ \Gamma \ts T_2 \hasnt l}
{\Gamma \ts T_1 \tor T_2 \hasnt l}

\end{multicols}
}
\end{figure}

% TODO stable_typ restriction !!!

%%%%%%%%%%%%%%%%%%%%%

\begin{figure}
  \figurebox{
\pdfbookmark[0]{Well-formedness}{wf}
\begin{multicols}{2}[\judgement{Well-formed types}{\fbox{$\Gamma \ts T \wf$}}]

      \infax[\textsc{TODO}]
      {}
      
%Inductive wf_typ_impl: ctx -> fset typ -> typ -> Prop :=
%  | wf_top: forall G A,
%      wf_typ_impl G A typ_top
%  | wf_bot: forall G A,
%      wf_typ_impl G A typ_bot
%  | wf_hyp: forall G A T,
%      T \in A ->
%      wf_typ_impl G A T
%  | wf_rcd: forall G A D,
%      wf_dec_impl G (A \u \{(typ_rcd D)}) D ->
%      wf_typ_impl G A (typ_rcd D)
%  | wf_sel: forall G A x X L T U,
%      binds x X G ->
%      stable_typ X -> (* <-- important restriction *)
%      typ_has G X (dec_typ L T U) ->
%      wf_typ_impl G A X ->
%      wf_typ_impl G A T ->
%      wf_typ_impl G A U ->
%      wf_typ_impl G A (typ_sel (avar_f x) L)
%  | wf_and: forall G A T1 T2,
%      wf_typ_impl G A T1 ->
%      wf_typ_impl G A T2 ->
%      wf_typ_impl G A (typ_and T1 T2)
%  | wf_or: forall G A T1 T2,
%      wf_typ_impl G A T1 ->
%      wf_typ_impl G A T2 ->
%      wf_typ_impl G A (typ_or T1 T2)
%  | wf_self: forall G A x X,
%      binds x X G ->
%      wf_typ_impl G A X ->
%      wf_typ_impl G A (typ_self (avar_f x))
%  | wf_ex: forall L G A T U,
%      (forall x, x \notin L -> wf_typ_impl (G & x ~ (open_typ x T)) A (open_typ x T)) ->
%      (forall x, x \notin L -> wf_typ_impl (G & x ~ (open_typ x T)) A (open_typ x U)) ->
%      wf_typ_impl G A (typ_ex T U)
%  | wf_skolem: forall L G1 G2 x A S T U,
%      (forall y, y \notin L ->
%         wf_typ_impl (G1 & x ~ (open_typ y U) & y ~ (open_typ y S) & G2) A T) ->
%      fv_typ T \c (dom (G1 & x ~ typ_ex S U & G2)) -> (* instead of "y \notin fv_typ T" *)
%      wf_typ_impl (G1 & x ~ typ_ex S U & G2) A T
%with wf_dec_impl: ctx -> fset typ -> dec -> Prop :=
%  | wf_tmem: forall G A L Lo Hi,
%      wf_typ_impl G A Lo ->
%      wf_typ_impl G A Hi ->
%      wf_dec_impl G A (dec_typ L Lo Hi)
%  | wf_mtd: forall G A m U V,
%      wf_typ_impl G A U ->
%      wf_typ_impl G A V ->
%      wf_dec_impl G A (dec_mtd m U V).

    \end{multicols}

\vspace{7cm}
    \linesep

    \begin{multicols}{2}[\judgement{Well-formed declarations}{\fbox{$\Gamma \ts D \wf$}}]
      \infrule[\textsc{wf-tmem}]
      {\Gamma \ts S \wf ~~~~ \Gamma \ts U \wf}
      {\Gamma \ts \Ldecl L S U \wf}

      \infrule[\textsc{wf-mtd}]
      {\Gamma \ts S \wf ~~~~ \Gamma \ts U \wf}
      {\Gamma \ts \mdecl m S U \wf}

    \end{multicols}
  }

\end{figure}

%%%%%%%%%%%%%%%%%%%%%

\begin{figure}
  \figurebox{

\pdfbookmark[0]{Subtyping}{subtyping}
\begin{multicols}{2}[\judgement{Subtyping}{\fbox{$\Gamma \ts S \sub T$}}]

\infrule[\textsc{$\sub$-refl}]
{\Gamma \ts T \wf}
{\Gamma \ts T \sub T}

\infrule[\textsc{$\sub$-$\Top$}]
{\Gamma \ts T \wf}
{\Gamma \ts T \sub \Top}

\infrule[\textsc{$\Bot$-$\sub$}]
{\Gamma \ts T \wf}
{\Gamma \ts \Bot \sub T}

\infrule[\textsc{$\sub$-rcd}]
{\Gamma \ts D_1 \sub D_2}
{\Gamma \ts \{D_1\} \sub \{D_2\}}

\infrule[\textsc{$\sub$-sel-l}]
{(x: T) \in \Gamma ~~~~~~~~~~~  \Gamma \ts T \wf \\
 \Gamma \ts T \ni (\Ldecl L S U) ~~~ \Gamma \ts S \sub U}
{\Gamma \ts x.L \sub U}

\infrule[\textsc{$\sub$-sel-r}]
{(x: T) \in \Gamma ~~~~~~~~~~~  \Gamma \ts T \wf \\
 \Gamma \ts T \ni (\Ldecl L S U) ~~~ \Gamma \ts S \sub U}
{\Gamma \ts S \sub x.L}

%%%%

\infrule[\textsc{$\sub$-$\tand$}]
{\Gamma \ts T \sub U_1 ~~~~ \Gamma \ts T \sub U_2}
{\Gamma \ts T \sub U_1 \tand U_2}

\infrule[\textsc{$\sub$-$\tand$-1}]
{\Gamma \ts T_1 \wf ~~~~ \Gamma \ts T_2 \wf}
{\Gamma \ts T_1 \tand T_2 \sub T_1}

\infrule[\textsc{$\sub$-$\tand$-2}]
{\Gamma \ts T_1 \wf ~~~~ \Gamma \ts T_2 \wf}
{\Gamma \ts T_1 \tand T_2 \sub T_2}

\infrule[\textsc{$\sub$-$\tor$}]
{\Gamma \ts T_1 \sub U ~~~~ \Gamma \ts T_2 \sub U}
{\Gamma \ts T_1 \tor T_2 \sub U}

\infrule[\textsc{$\sub$-$\tor$-1}]
{\Gamma \ts T_1 \wf ~~~~ \Gamma \ts T_2 \wf}
{\Gamma \ts T_1 \sub T_1 \tor T_2}

\infrule[\textsc{$\sub$-$\tor$-2}]
{\Gamma \ts T_1 \wf ~~~~ \Gamma \ts T_2 \wf}
{\Gamma \ts T_2 \sub T_1 \tor T_2}

\end{multicols}

\vspace{0.1em}

\begin{multicols}{2}

\infrule[\textsc{$\sub$-trans}]
{\Gamma \ts T_1 \sub T_2 \\ \Gamma \ts T_2 \sub T_3}
{\Gamma \ts T_1 \sub T_3}

\infrule[\textsc{$\sub$-hyp}]
{(x: T) \in \Gamma ~~~~  \Gamma \ts T \wf \\
 \Gamma \ts T \ni (\Ldecl L S U) \\
 \Gamma \ts S \wf ~~~~  \Gamma \ts U \wf}
{\Gamma \ts S \sub U}

\end{multicols}

\vspace{0.1em}

\newruletrue

\begin{multicols}{2}

\infrule[\textsc{$\sub$-self-l}]
{(x: T) \in \Gamma \\  \Gamma \ts T \wf}
{\Gamma \ts x\sing \sub T}

\infrule[\textsc{$\sub$-ex-l}]
{\Gamma \envplus{x: S} \ts T \sub U \\
 \Gamma \ts U \wf ~~~~ \Gamma \envplus{x: S} \ts S \wf}
{\Gamma \ts {\exTyp x S T} \sub U}

\infrule[\textsc{$\sub$-self-r}]
{(x: T) \in \Gamma \\
 \Gamma \ts x\sing \wf \\
 \Gamma \ts y\sing \wf}
{\Gamma \ts y\sing \sub x\sing}

\infrule[\textsc{$\sub$-ex-r}]
{(x: S') \in \Gamma ~~~~ \Gamma \ts S' \sub S \\
 \Gamma \ts T \sub U}
{\Gamma \ts T \sub {\exTyp x S U}}

\end{multicols}

\newrulefalse

\linesep

\begin{multicols}{2}[\judgement{Declaration subtyping}{\fbox{$\Gamma \ts D \sub D'$}}]

    \infrule[\textsc{subdec-typ}]
            {~~\Gamma \ts S' \sub S ~~~~ \Gamma \ts T \sub T'~~}
            {\Gamma \ts (\Ldecl L S T) \sub (\Ldecl L {S'} {T'})}

    \infrule[\textsc{subdec-mtd}]
            {\Gamma \ts S' \sub S ~~~~ \Gamma \ts T \sub T'}
            {\Gamma \ts (\mdecl m S T) \sub (\mdecl m {S'} {T'})}

\end{multicols}
}
\end{figure}


%%%%%%%%%%%%%%%%%%%%%

\begin{figure}
\figurebox{
\pdfbookmark[0]{Typing}{typing}
\begin{multicols}{2}[\judgement{Term Typing}{\fbox{$\Gamma \ts t : T$}}]

\infrule[\textsc{ty-var}]
{(x: T) \in \Gamma\\
 \Gamma \ts T \wf}
{\Gamma \ts x: T}

\infrule[\textsc{ty-call}]
{\Gamma \ts t : T ~~ \Gamma \ts u : U  ~~ \Gamma \ts V \wf \\
 \Gamma \ts T \ni (\mdecl m U V)}
{\Gamma \ts {\trmcall t m u} : V}

\infrule[\textsc{ty-new}]
{\Gamma \envplus{x: T} \ts \{\seq{d}\} : T \\
 \Gamma \envplus{x: T} \ts u : U ~~~~ \Gamma \ts U \wf}
{\Gamma \ts {\trmnew x {\seq{d}} u}: U}

\infrule[\textsc{ty-sbsm}]
{\Gamma \ts t: T_1\\
 \Gamma \ts T_1 \sub T_2}
{\Gamma \ts t: T_2}

\end{multicols}

\linesep

\begin{multicols}{2}[\judgement{Initialization Typing}{\fbox{$\Gamma \ts d : D$}}]

\infrule[\textsc{ty-tdef}]
{\Gamma \ts T \wf}
{\Gamma \ts (\deftyp L T) : (\Ldecl L T T)}

\infrule[\textsc{ty-mdef}]
{\Gamma \ts T \wf ~~~~ \Gamma \ts U \wf \\
 \Gamma \envplus{x: T} \ts u : U}
{\Gamma \ts (\defmtd m x T U u) : (\mdecl m T U) }

\end{multicols}

\linesep

\begin{multicols}{2}[\judgement{Initialization List Typing}{\fbox{$\Gamma \ts \{\seq{d}\} : T$}}]

\infrule[\textsc{ty-nil}]
{}
{\Gamma \ts \{\} : \Top}

\infrule[\textsc{ty-cons}]
{ label(d) \notin labels(\seq{d})\\
  \Gamma \ts \{ \seq{d} \} : T ~~~~ \Gamma \ts d : D }
{\Gamma \ts \{ \seq{d}, d \} : T \tand \{ D \} }

\end{multicols}
}

\end{figure}

%%%%%%%%%%%%%%%%%%%%%%%%%%%%%%%%%


\end{document}
